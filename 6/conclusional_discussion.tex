\chapter{Conclusions} %chapter 7
\label{chap:conclusions}

%This section summarize 
%	the exciting overview and research motivations
%	the works that has been done
%	the significant contribution of the work 
%	the intrigue and feasible of future research project to extend the current work.

%Small recap on what I have done
%Then list out the results with
%their implications
%their limitation
%Then introduce what may be done next
%That is all, finally!

This chapter concludes the work, summarizes main contributions and lists possible directions for future research.


\section{Contributions of the Thesis}
\label{sec:conclusion_contribution}

The research sets out to seek the solution for the dual-problem of 
1) establishing scalar and representative statistical models for the multi-dimensional and inter-correlated POLSAR data 
and 
2) developing novel additive and homoskedastic measures of distance for both SAR and POLSAR data, which allow the benefits of the homoskedastic statistical estimation framework to be demonstrated. 
After different scalar statistical models, together with many different data processing techniques, is reviewed for both SAR and POLSAR data,
  chapter \ref{chap:lit_survey} highlights the needs for scalar, representative and homoskedastic statistical models for the multi-dimensional, complex and heteroskedastic data.

  Chapter \ref{chap:sar} then presents homoskedastic models that are derived for heteroskedastic SAR data. 
In chapter \ref{chap:polsar}, scalar and representative statistical models are proposed for the multi-dimensional and inter-correlated POLSAR data.
These two seemingly disconnected results are unified, when the statistical models for POLSAR are shown to be the generic multi-dimensional extension of the commonly used statistical models for SAR intensity.
This combination give rise to the main result: the proposal of scalar, representative and homoskedastic models for both SAR and POLSAR data.
These theoretical results are applied in chapter \ref{chap:applications} where a new clustering algorithm and a new SAR speckle filter are proposed together with our proposal of using MSE to evaluate both SAR and POLSAR speckle filters.

%What are the results?
Generally speaking, this thesis contributes solutions for the problem of how to interpret and process the multidimensional and stochastic (POL)SAR data.
%One way of contribution is to derive new and better techniques to help alleviate the situation.
%The most direct way of contributing to this problem is to
%as well as derives new and
%better techniques to process the data. %help alleviate the situation. 
%An even better way, however, is to derive new theoretical models from which novel data processing technique can be derived.
%In this thesis, new theoretical models are derived for the (POL)SAR data
%  in addition to several novel (POL)SAR data processing techniques
%  which are also proposed as beneficial applications of these models.
Specifically, the
                main theoretical results obtained in this thesis are novel statistical
                models, which subsequently lead to derivation of
                consistent and homoskedastic measures of distance, for
                both SAR and POLSAR data.
Compared
                to other scalar models for POLSAR the proposed models
                are highly representative of the multi dimensional data.
Its
                representative power is demonstrated as it lead to
                consistent discrimination measures as well as it
                generalizes the traditional representative model for SAR
                intensity towards the multidimensional domain.

%It is common-consensus that in the original multiplicative SAR domain,
%  the intensity ratio serves as the best available discrimination measure.
%Meanwhile, there are several proposed discrimination measures for POLSAR
%  but there are no apparent link between the two fields.
%This thesis shows that the logarithmic transformation not only converts multiplicative SAR into 
The proposal of using the determinant ratio as a POLSAR discrimination measure is also a novel contribution.
It can be considered as the extension of the SAR intensity ratio,
  which have long been considered as the best discrimination measures for traditional SAR data \cite{Rignot_1993_TGRS_896}.
It works consistently in the original multiplicative and heteroskedastic POLSAR domain,
  and the logarithmic transformation converts this ratio into a consistent subtractive measures of distance in the transformed additive and homoskedastic domain.
This
                derived consistent measures of distance for SAR and
                especially for POLSAR are also a novelty.
With regards to other existing discrimination measures for SAR and POLSAR,
                this thesis unite them, together with the new ones
                proposed herein, all into a consistent theory.
%More
%                over the homoskedastic models and their derived
%                consistent measures of distance bring about the benefits
%                of homoskedastic statistical estimation framework.

%The scalar POLSAR models are representative and lead to several consistent discrimination measures.
%This should be beneficial to a range of problems where single scalar number is required to represent complex multi dimensional POLSAR.
%These include. for example, target detection, surface cover classification, edge detection or POLSAR  speckle filters evaluations, etc 

%The scalar models for POLSAR are also shown to be the generic multi-dimensional models 
%which also encompass the traditional one-dimensional SAR intensity models under its natural coverage.
%Thus the additive and homoskedastic models initially developed for SAR are shown extensible to the generic multi-dimensional, multiplicative and heteroskedastic case of POLSAR.

In \textbf{summary}, to address the dual-problem stated earlier, this thesis proposes several
scalar, additive and homoskedastic models for the multi-dimensional, multiplicative
and heteroskedastic POLSAR data. 
The dis-similarity measures are computed for the determinant of the POLSAR covariance matrix,
  which when converted into one-dimensional data
  is gracefully transformed into the traditional SAR intensity.
The models are shown to be theoretically powerful.
Not only they include the one-dimensional SAR under their natural coverage 
they are also able to provide alternative and simpler explanations to a range of theoretical concepts. 
Consequently, these measures of dis-similarity may be employed in a wide range of applications
  where a scalar number is required to represent the complex multi-dimensional POLSAR data.
The proposed theoretical models are also versatile in practice
  as they cope well with imperfect conditions evident in real-life captured data.

\section{Discussion of the proposed models}

Let us consider the different theoretical properties of the proposed models.
First, the use of covariance matrix log-determinant may be related to the standard eigen-decomposition method of the POLSAR covariance matrices.
In fact, the log-determinant can also be computed as the sum of log-eigenvalues.
Specifically $\ln{|M|} = \sum \ln{\lambda_M}$ where $\lambda_M$ denotes all the eigenvalues of M.
Thus similar to other eigenvalue based approach (e.g. entropy/anisotropy, ...),
  the models presented here are invariant to polarization basis transformations.

Second, the models are developed for the POLSAR covariance matrix.
However, since the POLSAR coherency matrix is related to the covariance matrix via a unitary transformation which preserves the determinant,
  the model should also be applicable to the coherency matrix.

It should be noted that the models are far from complete.
% It reduces the multi-dimensional POLSAR data to a scalar value.
%While this is probably desirable for a wide range of application,
%  such a reduction is unlikely to be lossless.
While it is desirable to reduce the multi-dimensonal POLSAR data to a scalar value for many applications,
  such a reduction is unlikely to be lossless.  
Thus %to better understand POLSAR data
the use of this technique could be complemented with some high-dimensional POLSAR target-decomposition techniques, such as the Freeman Durden decomposition \cite{Freeman_1998_TGRS_963} or the entropy/anisotropy decomposition \cite{Cloude_1997_TGRS_68} or similar.

Nevertheless the proposed models are promising.
Even though initially developed for partial and monostatic POLSAR data,
  it was shown to be applicable to traditional SAR data as well.
Since the models assumptions are quite minimal, they may potentially applicable to bi-static and interferometric data, although that would require further study.

%If move to chapter 4, can add these paragraphs
%The thesis also derives the homoskedastic models for both SAR and POLSAR heteroskedastic data.
%Within this consistent variance models, several consistent sense of subtractive distance is proposed.
%These can be thought of as the linear and subtractive versions of the commonly used ratio discrimination measure in the original multiplicative SAR model.
%
%Theses additive and homoskedastic models should be beneficial to all the steps of the computational and statistical estimation framework in particular as well as to many existing digital technologies and algorithms in general.
%This thesis also includes several published studies which illustrate several of such benefits.
%These include the use of the consistent variance to break the vicious circle of speckle filtering, as well as the use of MSE in evaluating (POL)SAR Statistical estimators within the additive and homoskedastic model.
%

 \subsection{Beneficial implications of the proposed models}

%How to exploit the advantages of these new results? What
%                are their benefit? And
%                to whom? OK, assuming your solution is novel and it can
%                hold water, is it useful? How is it better than the
%                state-of-the-art? And to whom does this matter? Why
%                should they use your solutions instead of other?
The over-arching strategic benefit is to enable the application of existing and established techniques from one matured field into another less-than-developed new field.
An example is between the fields of SAR and POLSAR, where obviously the SAR data is studied in much further details than the newly derived POLSAR data.
%As is shown, the one-dimensional SAR neatly falls into the limit special case of the multi-dimensional POLSAR.
By putting both SAR and POLSAR into one consistent theory, this thesis opens up the opportunities to apply many of the existing SAR processing technique to POLSAR.
Another example is between the fields of computer science and SAR-based remote sensing.
Evidently while the number of different digital image processing, computational intelligence or signal processing techniques for digital input data is many times higher than the number of available techniques applicable to (POL)SAR data.
And the proposed homoskedastic models, together with the consistent measures of distance,  %a solution for the second problem
could potentially allows this currently small portion to grow larger.
%a plethoria number of digital image processing algorithms as well as artificial intelligence and machine learning techniques to be applicable for both SAR and POLSAR data.

%Speaking from the author's experience since he first being introduced to this field, the solution for these problem should be important for the next generation of new-comers to the field of (POL)SAR.
%This should also be interesting for experienced experts in the field %as it will increase our capabilities in understanding and processing of (POL)SAR data.
The proposed homoskedastic statistical estimation framework provides a firm foundation
to overcome the various negative impacts of heteroskedasticity on statistical estimations.
This
                is demonstrated in both stages of developing new
                statistical estimators (e.g. speckle filters) as well as evaluating their estimating performance.
Specifically
                in the context of SAR speckle filters: consistent
                variance and homoskedasticity is shown to help in
                breaking the vicious circle of speckle filtering.

At
                the same time, a consistently meaningful sense of
                distance leading to our proposal of using the familiar
                MSE in the homoskedastic domain to evaluate SAR speckle
                filters.
Intuitively,
                this comes about as the speckle suppression power of
                speckle filters can be evaluated via variance estimation
                practice, while the radiometric preservation can be
                evaluated via the normal bias evaluation.
Combined,
                while there are many different evaluation criteria in
                the heteroskedastic SAR speckle filter evaluation
                framework, the MSE is proposed as the single unified
                evaluation criteria for SAR speckle filters.
In
                short, it is shown that in the homoskedastic domain, not
                only it is possible to find the equivalent of existing
                methods overcoming heteroskedastic drawbacks in the
                original domain, it is also possible to suggests further
                improvements based on existing techniques.

The
                proposed models for POLSAR also bring forward an
                important benefits.
Since
                the models, especially those for discrimination
                measures, can be considered as the multidimensional
                extension of the models for SAR, it makes many existing
                SAR processing techniques extensible towards POLSAR
                data.
The POLSAR determinant ratio, for example, can be considered as
                the extension of the ubiquitous SAR intensity ratio.
This
                thesis also illustrates how the MSE evaluation
                criteria can also be extended from SAR to POLSAR.

From another angle, the theoretical models for POLSAR also offer several benefits.
The theoretical models may also provide an alternative derivation for the widely used likelihood test statistics in POLSAR.
In view of the models given in Eqns \ref{eqn:determinant_distribution} \& \ref{eqn:log_determinant_distribution},
  the likelihood test statistics exposed in \cite{Conradsen_2003_TGRS_4} and rewritten in Eqns \ref{eqn:Complex Wishart Distribution Likelihood Test Statistics} \& \ref{eqn:Complex Wishart Distribution Likelihood Test Statistics (Log Domain)}
can be simulated as:
\begin{align*}
  \ln{Q} &\sim  k + L_x \Lambda^d_{L_x} + L_y \Lambda^d_{L_y} - (L_x + L_y) \Lambda^d_{(L_x + L_y)} \\
  Q &\sim e^k \frac{(\chi^d_{L_x})^{L_x} \cdot (\chi^d_{L_y})^{L_y}}{(\chi^d_{L_x + L_y})^{L_x + L_y}}   
\end{align*}
where $k = d \left[ (L_x + L_y) \ln(L_x + L_y) - L_x \ln{L_x} - L_y \ln{L_y} \right]$.
As a by-product of this exact derivation, the current paper also proposed several simpler discrimination measures for the common case of $L_x=L_y$.

Similar to the way that other measures of distance can be used to derive POLSAR classifiers \cite{Lee_1999_TGRS}, change detectors \cite{Conradsen_2003_TGRS_4}, edge detectors \cite{Schou_2003_TGRS_20} or other clustering and speckle filtering techniques \cite{Le_2010_ACRS} \cite{Le_2011_ACRS}, 
new detection, classification, clustering or speckle filtering algorithms can be derived using the models proposed.

%While some of these obstacles have been overcome by various experienced researcher in the field proposing different genuine approaches, 
%  the additive and homoskedastic transformation approach proposed in this thesis suggests that these problems also have equivalent handling techniques which usually are more familiar.
%Furthermore competitive and even sometimes better proposal are made making use of this familiarity.
%For example, not only it is shown that both radiometric preservation and speckle suppression evaluation of SAR speckle filters can be carried out equivalently well under both domain, 
%  MSE is in the additive and homoskedastic domain is shown capable of combining the two evaluations, which so far have not been figured out in the original heteroskedastic domain.
%Elsewhere, the application of wavelet in this same transformed domain has resulted in a new class of speckle filters which performs respectably in comparison to other well established filters.
%The familiarity of this transformed domain should also be very helpful for the new comers of the field, who is much more likely to find comfort ability in this derived domain in comparison to the original multiplicative and heteroskedastic domain. 

\subsection{Limitations of the proposed models}

%What are their limitations? What problems this appears
%                to solve, but actually is not?

The
                proposed scalar models for POLSAR also have limitations.
It
                calls for a dimensional reduction operation, which is
                probably not lossless.
Thus
                while the determinant of POLSAR covariance matrix is
                said to be "highly representative", it is definitely not
                fully representative of the data.
Intuitively,
                this is similar to the use of magnitude as being
                representative of a complex number, which also is not
                fully representative of the two-dimensional number.
But it is important to emphasize that,
                just like the magnitude is invariant to the rotation of
                the reference frame, the POLSAR determinant is also
                invariant to polarization basis transformation.

While
                the proposed homoskedastic measures of distance have
                their own benefits, their biggest challenge probably
                lies in changing the hard learned lessons that
                previous generation of researchers have experienced tackling
                the multiplicative and heteroskedastic SAR data.
At
                the same time, while the logarithmic transformation has
                the advantages of converting the multiplicative and
                heteroskedastic model into an additive and homoscedastic
                data, such transformation does not arrive conveniently
                to the common additive white noise model.
In fact, figures presented in the previous chapters show that 
	they are not even centred around the origin. 
This may explain why averaging filters in the log-transformed domain (e.g. \cite{Arsenault_JOptSocAm_1976}) do not 
	work very well in practice, as the operation is biased and statistically inconsistent.
To counter this, the use of maximum likelihood estimation, instead of simple averaging, 
	is suggested \cite{Le_2011_ACRS}.
While averaging is also the MLE operator in the SAR's original domain,
                the MLE operation in the log-transformed domain is admittedly more complex in comparison.

\section{Possible Future Work}
\label{sec:conclusion_future_work}

%What is still lacking to solve these problems? What is
%                the future work to extend this theory? What can be done
%                next?

While
                the proposed models may have certain limitations, its
                potentials still mostly stays undiscovered.
Since
                the assumptions of partial and full monostatic POLSAR
                models are quite minimal, they are probably also
                applicable to other types of SAR-based data such as: Interferrometric SAR (InSAR), Polarimetric Interferrometric SAR (POLINSAR) 
                or bistatic POLSAR data.
In addition, further experiments can be done to determine whether some existing SAR data processing techniques for SAR can be equally applicable to POLSAR data.
%                Also,
%                it may be interesting time to choose some existing data
%                processing techniques for SAR and showing them
%                applicable to POLSAR data.

In
                the same vein, it is probably even more significant to
                investigate the applicability of some certain existing
                additive and homoskedastic based computational data
                processing techniques towards the multiplicative and
                heteroskedastic (POL)SAR data.
As the additive noise PDF in log-domain has been worked out, we hypothesize that the true back-scattering coefficient PDF, also in log-domain, can be estimated through regularized deconvolution algorithms.
If an algorithm could be designed to estimate such PDF, we believe this would present an improvement for a whole class of speckle filters, namely the Maximum-A-Priori (MAP) class, which up to now, still assumes the true PDF will follow certain known fixed distributions.

Furthermore, as the point spread function (PSF) is available in the log-transformed domain, it appears that, for the first time, the histogram of the underlying back-scattering coefficient may become estimate-able. 
In Fig. 
	\ref{fig:noise_pdf_as_psf:conv_inputs}, 
the underlying single channel grey image histogram is presented along with the noise PDF, in the log-transformed domain. 
As the noise is additive, it is expected that the noise PDF will act as a point spread function.
This makes the observable speckled image histogram mathematically equivalent to the convolution of the original image's histogram and the noise PDF.
This characteristic is reflected clearly in Fig. 
	\ref{fig:noise_pdf_as_psf:conv_output_observable}, 
where except for the un-calibrated shifting, the shapes of theoretical convolution and observed histogram are reasonably similar.
We hence hypothesize that: given the observable SAR data and its histogram, it should be possible to estimate the original underlying signal's histogram through some regularized deconvolution methods,
  such as using the Richardson-Lucy algorithm \cite{Richardson_1972_JOptSocAm, Lucy_1974_JAstronomical}.
%We intended to start by applying the Richardson-Lucy algorithm \cite{Richardson_1972_JOptSocAm, Lucy_1974_JAstronomical} for such estimations.

\begin{figure}[h!]
\centering
	\subfloat[Original image's histogram and PSF]{
		 \epsfxsize=2.5in
		 \epsfysize=2.5in
 		 \epsffile{images/log_PSF_NTU_gray_histogram.eps}
		 \label{fig:noise_pdf_as_psf:conv_inputs}
	} 
	\hfill
	\subfloat[Convoluted and speckled image histogram]{
		 \epsfxsize=2.5in
		 \epsfysize=2.5in
		 \epsffile{images/log_convoluted_NTU_speckled_histogram.eps} 	
		 \label{fig:noise_pdf_as_psf:conv_output_observable}
	}
\caption{Observable convolution of PSF}
\label{fig:noise_pdf_as_psf}
\end{figure}
  
For multi-channel SAR data, the short-term focus is to extend the existing statistical estimation framework for SAR towards POLSAR.
Specifically, the F-MLE speckle filter for SAR should be extensible towards POLSAR,
Similarly, the study to simulate POLSAR data as well as to use MSE to evaluate POLSAR speckle filters is also highly achievable.
Lastly, the representative models for POLSAR may also be further explored, such as to explain if the additional channels provided in POLSAR do indeed lead to improved information being captured.  
%Another way of extension is to investigate other types of multi-channel SAR like bi-static POLSAR or InSAR.
       
%\section{Conclusion}
%\label{sec:conclusion_final}
%
%The thesis sets out to provide the solution for the dual-problem of
%1) establishing scalar and representative statistical models for the multi-dimensional and inter-correlated POLSAR data
%and 2) developing novel additive and homoskedastic measures of distance for both SAR and POLSAR data which allow the benefits of the homoskedastic statistical estimation framework to be demonstrated.
%In chapter \ref{chap:sar}, this thesis presents homoskedastic models are derived for heteroskedastic SAR data.
%Subsequently in chapter \ref{chap:polsar}, scalar and representative statistical models are proposed for the multi-dimensional and inter-correlated for POLSAR data.
%These two seemingly disconnected results are unified, when the statistical models for POLSAR are shown to be the generic multi-dimensional extension of the commonly used statistical models for SAR intensity.
%These theoretical results are applied in chapter \ref{chap:applications} where a new clustering algorithm and a new SAR speckle filter are proposed together with our proposal of using MSE to evaluate both SAR and POLSAR speckle filters.
%
