\documentclass{beamer}

\usepackage{default}

\begin{document}

\underline{\textbf{1. Introduction}}

\textbf{  1.1. The purpose of this writing (for what type of audiences?)}

For every one, including me to understand!

\textbf{  1.2. Describe the given context}

Given the pdf of $L$ is
\begin{equation}
\label{eqn:1_1}
pdf(L) = e^{L-e^L}
\end{equation}
with $L \in (-\infty, \infty)$

\textbf{  1.3. The problem posed}

Find the population expected mean and variance of $L$

\textbf{  1.4. The divided structure and the campaign to explore and conquer}

First expected mean is calculated, then expected variance will follow.

\underline{\textbf{2. Divide and conquere campaign}}

\textbf{2.1 Expected mean of $L$}

Plug Eqn. (\ref{eqn:1_1}) into Eqn. (\ref{eqn:3_1}), it becomes
\begin{equation}
\label{eqn:2_1}
E(L) = \int_{-\infty}^{\infty} \! L \cdot e^{L-e^L} \, dL
\end{equation}

\textbf{2.1.1 Finding indefinite integral}

It is going to be proved that 
\begin{equation}
\label{eqn:2_2}
\int_L \! L \cdot e^{L-e^L} \, dL = Ei(-e^L)-e^{(-e^L)} L
\end{equation}

Bearing Eqn. (\ref{eqn:3_4}) in mind, consider
\begin{eqnarray*}
\frac{d}{dL} \left[ Ei(-e^L) \right] 
	&=& \frac{d \left[ Ei(-e^L) \right] }{d \left[ -e^L \right]} \cdot \frac{d \left[ -e^L \right]}{dL} \\
	&=& \frac{ e^{(-e^L)}  }{ -e^L } \cdot  \left[ -e^L \right] \\
\end{eqnarray*}

Thus
\begin{equation}
\label{eqn:2_3}
\frac{d}{dL} \left[ Ei(-e^L) \right]  = e^{(-e^L)}
\end{equation}

Bearing Eqn. (\ref{eqn:3_3}) in mind, consider
\begin{eqnarray*}
\frac{d}{dL} \left[ e^{(-e^L)} L \right]  
	&=& e^{-e^L} \frac{dL}{dL} + L \frac{d \left[ e^{-e^L}  \right]}{dL} \\
	&=& e^{-e^L} + L \frac{d \left[ e^{-e^L}  \right]}{d \left[ -e^L  \right] } \cdot \frac{d \left[ -e^L  \right]}{dL} \\
	&=& e^{-e^L} - L  e^{-e^L}   e^L  \\
\end{eqnarray*}

Thus
\begin{equation}
\label{eqn:2_4}
\frac{d}{dL} \left[ e^{(-e^L)} L \right]  = e^{-e^L} - L  e^{L-e^L}
\end{equation}

Let Eqn. (\ref{eqn:2_3}) - Eqn. (\ref{eqn:2_4}), Eqn. (\ref{eqn:2_2}) becomes evident
\begin{eqnarray*}
\frac{d}{dL} \left[ Ei(-e^L) -  e^{(-e^L)}  L \right]
	&=& e^{(-e^L)} - \left( e^{-e^L} - L  e^{L-e^L} \right) \\
	&=& L  e^{L-e^L} \\
\end{eqnarray*}

\textbf{2.1.2 Estimating the mean}

Plug Eqn. (\ref{eqn:2_2}) into Eqn(\ref{eqn:2_1}), it becomes
\begin{equation}
\label{eqn:2_5}
E(L) = \left[ Ei(-e^L)-e^{(-e^L)} L \right]_{-\infty}^{\infty}
\end{equation}

Consider 
\begin{eqnarray*}
\lim_{L \to \infty}  {Ei(-e^L)-e^{(-e^L)} }
	&=& \lim_{L \to \infty} {Ei(-e^L)} - \lim_{L \to \infty}{e^{(-e^L)} L} \\
	&=& Ei(-\infty) - \lim_{L \to \infty}{ \frac{L}{e^{(-e^L)}} } \\
	&=& 0 - \lim_{L \to \infty}{ \frac{1}{ \frac{d e^{(-e^L)}}{dL} } } \\
\end{eqnarray*}

Thus
\begin{equation}
\label{eqn:2_6}
\lim_{L \to \infty}  {Ei(-e^L)-e^{(-e^L)} } = 0
\end{equation}

Consider 
\begin{eqnarray*}
\lim_{L \to -\infty}  {Ei(-e^L)-e^{(-e^L)} L }
	&=& \lim_{L \to -\infty} {Ei(-e^L)} - \lim_{L \to -\infty}{e^{(-e^L)} L} \\
	&=& \lim_{L \to -\infty} {Ei(-e^L)} - e^{ \left(-\displaystyle{ \lim_{L \to -\infty}{e^L} } \right)} L   \\
	&=& \lim_{L \to -\infty} {Ei(-e^L)} -  L   \\
\end{eqnarray*}

Evaluating is not straight forward, (Wolfram Alpha gives no step, just result)
\begin{equation}
\label{eqn:2_7}
\lim_{L \to -\infty}  {Ei(-e^L)-e^{(-e^L)} } = \gamma
\end{equation}

Plug Eqns. (\ref{eqn:2_6}) and (\ref{eqn:2_7}) into Eqn(\ref{eqn:2_5}), it becomes
\begin{equation}
\label{eqn:2_8}
E(L) = - \gamma
\end{equation}

\textbf{2.1.3 Alternative method}

Set
\begin{eqnarray*} 
 z &=& e^L \text{ then } L = \ln(z) \text{ and } dz = e^L dL \\
E(L) 
	&=& \int_{0}^{\infty} \! L \cdot e^{L-e^L} \, dL \\
	&=& \int_{0}^{\infty} \! L \cdot e^{-e^L} e^L \, dL \\
	&=& \int_{0}^{\infty} \! \ln(z) \cdot e^{-z}  \, dz \\
\end{eqnarray*}
Thus
\begin{equation}
\label{eqn:2_9}
E(L) = - \gamma
\end{equation}


\textbf{2.2 Expected variance of $L$}

Bearing Eqns. (\ref{eqn:1_1}) and (\ref{eqn:2_8}) in mind, Eqn. (\ref{eqn:3_2}) can be written as:
\begin{eqnarray*}
 var(L) 
	&=& \int_L \! (L-E(L))^2 \cdot e^{L-e^L}  \, dL \\
	&=& \int_{-\infty}^{\infty} \! (L-\gamma)^2 \cdot e^{L-e^L}  \, dL \\
\end{eqnarray*}

Thus
\begin{equation}
\label{eqn:2_10}
var(L) 	= \int_{-\infty}^{\infty} \! L^2 \cdot e^{L-e^L}  \, dL  
		+ \int_{-\infty}^{\infty} \! \gamma^2 \cdot e^{L-e^L}  \, dL  
		- \int_{-\infty}^{\infty} \! 2 \gamma L \cdot e^{L-e^L}  \, dL  
\end{equation}

Consider 
\begin{eqnarray*}
\int_{-\infty}^{\infty} \! e^{x-e^x}  \, dx  
	&=& \left| -e^{-e^x} \right|_{-\infty}^{\infty} \\
	&=& \lim_{x \to \infty} {-e^{-e^x}} - \lim_{x \to -\infty} {-e^{-e^x}} \\
	&=& -e^{-\infty} + e^{0} 
\end{eqnarray*}

Thus
\begin{equation}
\label{eqn:2_11}
\int_{-\infty}^{\infty} \! e^{x-e^x}  \, dx  = 1
\end{equation}

Set
\begin{eqnarray*} 
 z &=& e^L \text{ then } L = \ln(z) \text{ and } dz = e^L dL \\
\int_{-\infty}^{\infty} \! L^2 \cdot e^{L-e^L}  \, dL 
	&=& \int_{-\infty}^{\infty} \! L^2 \cdot e^{e^L} e^L \, dL \\
	&=& \int_{0}^{\infty} \! \ln(z)^2 \cdot e^{z}  \, dz \\
	&=& (-1)^2 \int_{0}^{\infty} \! (\ln z)^2 \cdot e^{z}  \, dz 
\end{eqnarray*}

Plug in Eqns. (\ref{eqn:3_5}) and (\ref{eqn:3_6}), this becomes
\begin{equation}
\label{eqn:2_12}
\int_{-\infty}^{\infty} \! L^2 \cdot e^{L-e^L}  \, dL = I_2 = \gamma^2 + \frac{\pi^2}{6}
\end{equation}

Plug Eqns. (\ref{eqn:2_12}), (\ref{eqn:2_11}) and (\ref{eqn:2_9}) into Eqn. (\ref{eqn:2_10}), it becomes
\begin{equation}
\label{eqn:2_13}
var(L) = \gamma^2 + \frac{\pi^2}{6} + \gamma^2 - 2 \gamma^2 = \frac{\pi^2}{6}
\end{equation}

\underline{\textbf{3. Evaluation and Discussion}}

\textbf{3.1. Assumptions Used:}

Population expected mean calculation
\begin{equation}
  \label{eqn:3_1}
E(x) = \int_x \! x \cdot pdf(x)  \, dx
\end{equation}

Population expected variance calculation
\begin{equation}
  \label{eqn:3_2}
var(x) = \int_x \! (x-E(x))^2 \cdot pdf(x)  \, dx
\end{equation}

Derivative Product Rule
\begin{equation}
  \label{eqn:3_3}
\frac{d}{dx} (u v) = v \cdot \frac{du}{dx}+u \cdot \frac{dv}{dx}
\end{equation}

Exponential Integral
\begin{equation}
  \label{eqn:3_4}
Ei(x) = \int_{- \infty}^x \! \frac{e^t}{t} \, dt
\end{equation}

Euler-Mascheroni Integrals
\begin{equation}
  \label{eqn:3_5}
I_n = (-1)^n \int_0^\infty (\ln z)^n e^{-z} \, dz
\end{equation}

Particular values are
\begin{eqnarray*}
  \label{eqn:3_6}
I_0 	&=& 1 \\
I_1 	&=& \gamma \\
I_2 	&=& \gamma^2 + \frac{\pi^2}{6} \\
\end{eqnarray*}

\textbf{3.2. Heuristics and Ad-hoc Choices made: }

\underline{\textbf{4. Conclusion}}

\textbf{4.1. Assert purpose achieved, problem conquered}

The mean and variance are

\begin{equation}
\label{eqn:4_1}
avg(L) = \gamma
\end{equation}

\begin{equation}
\label{eqn:4_2}
var(L) = \frac{\pi^2}{6}
\end{equation}

\textbf{4.2. Lessons learned and its significance}

It can be seen that Exponential Integral and Euler-Mascheroni Integrals are related!

\begin{equation}
\label{eqn:3_7}
E(L) = - \gamma = -I_1 = - \lim_{L \to -\infty}  {Ei(-e^x)- x }
\end{equation}

\end{document}
