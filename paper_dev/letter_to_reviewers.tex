\documentclass{letter}
\usepackage{hyperref}
\signature{The Authors}
\address{}
\begin{document}
 
\begin{letter}{The Reviewers,\\IEEE Transaction on Geoscience and Remote Sensing}
\opening{Dear Reviewers,}

%\documentclass{article}
%\title{Cover letter to the Reviewers}

%\begin{document}

%\maketitle


We appreciate your time and effort in reviewing our work.
This paper presents novel statistical models for the multi-dimensional POLSAR data which generalize the commonly used statistical model for the one-dimensional SAR data and thus provides a bridge between the two fields inviting the adaptation of many existing SAR data processing techniques for POLSAR data.
We hope that, like us when we were faced with its results,
  you will find satisfaction as well as excitement after reading through this paper.
We appreciate your feedback in whatever way it may be in making this manuscript to become a better publication.
To facilitate your review, we enclose here a short synopsis of the paper.

We are aware that the paper is a bit too long.
Specifically, the mathematical derivations in the Appendices is presented in great details.
While this is supposed to expedite your review,
  it probably should be shortened in publication.
We would appreciate it if you can help us by pointing out which portion of this paper may be considered redundant by the TGRS readers. 

Once again, we thank you for your reviewing time and effort.

\closing{Our best regards}

\encl{The Synopsis of the Paper (next page)}

\newpage
\textbf{The Research Context:}

The extension from SAR to POLSAR brings about an issue that needs to be addressed:
  there exists not one, like the intensity in SAR, but many scalar observable quantities in POLSAR. 
While different statistical models have been developed for different POLSAR observables, 
  for a model to be useful, discrimination measures need to be derived.
Many POLSAR data processing techniques requires a scalar, representative observable and statistical model.

\textbf{Problem Statement:}

Practical application for POLSAR data processing requires the measures to be scalar, consistent and preferably homoskedastic on one hand.
On the other hand, the observable quantity being modelled need to be naturally representative for the high-dimensional POLSAR data.

\textbf{Newly Proposed Technique to Handle this Situation:}

In this paper, a couple of scalar statistical models for the determinant of the POLSAR covariance matrix are proposed.
These models are shown to be both representative and subsequently lead to statistically consistent discrimination measures.

\textbf{Shortcomings of the current POLSAR statistical models:}

There are a few published scalar statistical models for POLSAR, but none of them fit the dual criteria of being representative and at the same time lead to scalar discrimination measures.
There are also a few POLSAR discrimination measures, but all of them are based on the likelihood test statistics, which so far have only shown to be based on asymptotic distribution.

\textbf{Advantages of the newly proposed model}

Compared to other statistical models for POLSAR, the proposed models are scalar, and representative for POLSAR data.
Moreover, several multiplicative discrimination measures and additive measures of distance are derived from the proposed statistical models.
These POLSAR discrimination measures, for e.g. the determinant-ratio, can be considered as the multi-dimensional extension of the commonly used discrimination measures for the one-dimensional SAR intensity.
Compared to other discrimination measure proposals, this paper suggests an exact distribution, as opposed to asymptotic distributions, for the POLSAR likelihood statistical test.
As a by-product of this exact modeling, simpler discrimination measures, i.e. determinant-ratio and contrast, are proposed for the common case of Lx=Ly.

\textbf{High-level Summary of this work}

This paper presents novel statistical models for the multi-dimensional POLSAR data which generalize the commonly used statistical model for the one-dimensional SAR data and thus provides a bridge between the two fields inviting the adaptation of many existing SAR data processing techniques for POLSAR data.

\end{letter}
 
\end{document}
