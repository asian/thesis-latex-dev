%: ----------------------- introduction file header -----------------------
\chapter{Introducing The Research Question}

% the code below specifies where the figures are stored
%\ifpdf
%    \graphicspath{{1_introduction/figures/PNG/}{1_introduction/figures/PDF/}{1_introduction/figures/}}
%\else
%    \graphicspath{{1_introduction/figures/EPS/}{1_introduction/figures/}}
%\fi

% ----------------------------------------------------------------------
%: ----------------------- introduction content ----------------------- 
% ----------------------------------------------------------------------

Univariate SAR and multi-variate POLSAR data are multiplicative and heteroskedastic.
By definition, heteroskedastic noised data has its variance, and correspondingly its substractive distances,
  dependent on the main signal.
In another line of research, log transformation has been shown
  to convert the multiplicative nature of SAR data into a more familiar additive nature.
We hypothesize that   
  log-transformation also offers several consistent measures of distance in heteroskedastic SAR and POLSAR data.
This thesis aims to investigate and validate that hypothesis.

In this chapter: the main research problem and the corresponding hypothesis are explicitly stated.
The methodology section explains how the results are obtained and analyzed.
Contributions to advance theoretical knowledge are illustrated.
Towards the end of the thesis, an application illustrate how the theoretical results can be applied in future research and development activities.
  
%: ----------------------- HELP: latex document organisation
% the commands below help you to subdivide and organise your thesis
%    \chapter{}       = level 1, top level
%    \section{}       = level 2
%    \subsection{}    = level 3
%    \subsubsection{} = level 4
% note that everything after the percentage sign is hidden from output

	\section{Background Context and the Motivation to Study}

%Advantages of SAR in remote sensing paragraph (page 3)
Synthetic Aperture Radar can soon become the preferred choice for
  continous and autonomous large-scale remote surveillance solutions.
Compared to optical remote sensors, which is passive,
  active SAR sensors provides weather independent and night-inclusive operational capabilities.
Compared to other radar based active sensors, for example Real Aperture Radar (RAR),
  SAR provides a much better resolution, and larger coverage solution.
The once-limiting drawback of SAR is the high-demanding signal processing procedures,
  which is needed to process SAR images, has become obsolete,
  due to recent and quick advances in computational power  \citep{Cumming_2005_Artech}.

EM wave polarization and polarimetric SAR paragraph (page 3)

A paragraph about partial polsar and the extension from (page 3)

Problem: speckle in SAR and POLSAR, especially its heteroskedastic nature. (new)

The motivation then is to provide a probably better understanding about the statistics of speckle. (new)

For SAR data (page 4)

For partial POLSAR data (new)

For full polsar data (new)

The purpose of this section is to illustrate the importance of a computational framework needed in the interpretation and processing targets' stochastic polarimetry signatures. 
%This section start by describing the recent focus of space technology in earth observational remote sensing.
As reviewed in the second chapter, Synthetic Aperture Radar, i.e. SAR, offers a higher resolution, large coverage, weather-independent and night-capable imaging solution.
Recent advances in computational power has made the once-limiting high-demanding signal-processing procedures, which are needed to process SAR images, obsolete \cite{Cumming_2005_Artech}.
Thus we believe, SAR will soon come to be the preferred choice for continuous and autonomous large-scale remote surveillance solutions.

Electromagnetic waves have the intrinsic nature of polarization.
Polarimetric SAR, i.e. POLSAR, extends the SAR acquisition information from the traditional single SAR channel to multi-channel polarimetric SAR data, corresponding to different combination of transmitted and received polarizations.
Early works from the radar community had focused on the theoretical introduction of scattering matrix, i.e. Sinclair matrix or Mueller matrices, for full, i.e. quad, polarimetric measurements \cite{Sinclair_1950_ProcsIRE}.
%(Sinclair, 1950), (TODO) 
These works were continued by Boerner, and the first polarimetric images were captured by NASA JPL at Caltech in 1991 \cite{Zebker_1991_ProcsIEEE}.

Systems were then engineered and commercial POLSAR satellites, e.g. RadarSat, TerraSAR, Alos-PalSAR, and so on, were launched.
However due to operational restrictions on payload, power and data-downlink, the satellites also offer dual-channel polarimetric SAR modes apparently as a compromise between theoretical polarimetric requirements and practical limitations.
Initially to the end user, while dual-channel partial polarimetry may not provide the complete polarimetric signature of the target, the loss is compensated for by either better resolutions or higher coverage and lower cost.
Until very recently, these partial polarimetry modes were not formally taken up by scientists and academia. 
In his pioneer paper in 2005, Souyris\cite{Souyris_2005_TGRS} introduced the so-called compact polarimetry mode with the suggestion that it may be possible to reconstruct full polarimetric information from compact polarimetric data.
Raney\cite{Raney_2006_IGARSS} showed that dual-channel polarimetry is not only capable of measuring the complete polarization information of the back-scattered EM wave, via Stokes parameters, but also capable of detecting the basic target's polarimetric characters, i.e. odd or even bounce phenomena.

The extensions from single-channel SAR to multi-channel polarimetry can be considered as analogous to the advancement towards colour TV from the black-and-white days.
While the future is promising, both SAR and POLSAR suffer from speckle phenomena, which hampers the human capabilities to understand SAR images.
Speckle, which arises due to the random interference of many de-phased but coherent back-scattered waves, is inherent in all channels.
The effects of speckle is illustrated in Figure \ref{fig:sar_speckle_wrt_optical}.

%%TODO: SWITCH IMAGES: 24 lines
%\begin{figure}[h!]
%\centering
%	\subfloat[Optical Colour Image]{
%		 \epsfxsize=1.75in
%		 \epsfysize=1.75in
%		 \epsffile{images/20100502_mris_clipped.eps} 	
%		 \label{Original}
%	} 
%	\hfill
%	\subfloat[Optical Gray level image]{
%		 \epsfxsize=1.75in
%		 \epsfysize=1.75in
%		\epsffile{images/20100502_mris_clipped_gray.eps}
%		 \label{}
%	}
%	\hfill
%	\subfloat[Speckled SAR image]{
%		 \epsfxsize=1.75in
%		 \epsfysize=1.75in
%		 \epsffile{images/20100502_sar.eps} 	
%		 \label{Original}
%	} 
%\caption{Speckle phenomena in SAR data}
%\label{fig:sar_speckle_wrt_optical}
%\end{figure}

The motivation for this research then is to reconstruct high-quality grey-scale images from speckled single channel SAR data, and to visualise as colourful images the despeckled multi-channel POLSAR data.
In order to do so, the speckle effects need to be removed as much as possible from original (POL)SAR data.

Due to the random nature of speckle, speckle filtering has been put into the context of statistical estimation theory, which comprises statistical modelling and validation, estimator design and development as well as evaluation of different estimators' accuracy.
This proposed research project focuses on developing a computational framework for simulation, evaluation, processing and classification of target polarimetric signatures.
The benefits of such a framework are many fold.
Firstly, simulation allows one to quickly carry out experiments on small synthesized and well-designed scenarios, without the need for a real expensive satellite imagery.
Secondly, simulation provides ground-truth, without which quantitative evaluation of estimators would have been impossible.
Last but certainly not least, such a framework allow one to apply estimators repeatedly, and results from different estimators can be statistically and reliably compared.
 
For single-channel SAR images, our initial observations and experiences indicate that the community has been oblivious to many of the negative effects of statistical heteroskedasticity manifested in the original intensity and amplitude data.
In Chapter 4, it is shown that logarithmic transformation can convert SAR's multiplicative stochastic process into a better-understood additive noise model.
Thus, the additive-noise domain indicates homoskedasticity as well as offers a consistent sense of distance and variance.
As the additive noise PDF in log-domain has been worked out, we hypothesize that the true back-scattering coefficient PDF, also in log-domain, can be estimated through regularized deconvolution algorithms.
If an algorithm could be designed to estimate such PDF, we believe this would present an improvement for a whole class of speckle filters, namely the Maximum-A-Priori (MAP) class, which up to now, still assumes the true PDF will follow certain known fixed distributions.
%It is hoped that with the true PDF now being estimatable from the data itself, a better speckle filtering algorithm, in log-transformed domain, can be developed.

For multi-channel SAR data, as reviewed in Chapter 2, speckle filtering for the off-diagonal element $S_hS_v^*$ of the covariance matrix is still an open question.
Our observation has been that these elements are also available in dual-polarization SAR data, they specifically are part of Stokes parameters.
As mentioned earlier, Stokes parameters have been largely ignored in polarimetric literature, until recently asserted again by Raney in 2006 \cite{Raney_2006_IGARSS}.
Mathematically, Stokes parameters can be written as:
\begin{equation}
S = 
\left(
	\begin{array} {c}
		S_hS_h + S_vS_v \\
		S_hS_h - S_vS_v \\
		2 \Re{(S_hS_v^*)} \\
		2 \Im{(S_hS_v^*)}
	\end{array}
\right)
= 
\left(
	\begin{array} {c}
		S_hS_h + S_vS_v \\
		S_hS_h - S_vS_v \\
		S_+S_+ - S_-S_- \\
		S_lS_l - S_rS_r
	\end{array}
\right)
\end{equation}
We have evidence, and hypothesize that $S_+S_+$, $S_-S_-$, $S_rS_r$, $S_lS_l$ will have the same physical and statistical properties as the normal $S_hS_h$, $S_vS_v$.
If our hypothesis can be proven rigorously, it may then be possible to build and validate a statistical model for all elements in the covariance matrix of Polarimetric SAR data, namely $S_hS_h$, $S_vS_v$, $\Re{( S_hS_v^*)}$, $\Im{( S_hS_v^* )}$.
Also, speckle filtering on off-diagonal elements can be carried out using techniques, that were originally developed for single-channel SAR data, and up to now is only applicable to the diagonal elements of the covariance matrix.

For the topic of target (ship) detection and classification, we notice that the use of partial polarimetry will provide extra polarimetry data without incurring loss in spatial resolution.
We proceed to hypothesize that: partial polarimetry will lead to improved target (ship) detection and classification accuracy in comparision to traditional SAR data.
Experiments can be carried out by capturing polarimetric SAR remote sensing imagery and ground truth data which is provided via the ships' locational identification data.
Automatic Identification System (AIS), which normally supplements marine radars, is a standard ship tracking system for identifying and locating floating vessels.
Empirical evidence can be obtained by applying traditional ship detection and classification on traditional and polarimetric data concurrently, and the results are to be compared.
%Besides the empirical evidences that we plan to obtain, we hope that once our work on statistical modelling of partial polarimetry has taken shape, the hypothesis can be rigorously proven by statistical mathematics.

	\section{Objectives, Methodology and Contributions}

Copy the objectives (page 6)

copy the benefits (page 6)
        
\subsection{Objectives and Contributions}

The purpose of this section is to illustrate the objectives in developing computational frameworks for Stochastic Polarimetric Data. 
Assuming the framework will be implemented successfully, this section also describes its academic contributions as well as its projected practical benefits.
% for the industry.
%Also the academic contribution as well as practical benefits, assuming the frameworks is developed that meet the requirements, is to be described. 

%Even though the research topics appears segmentable into different sub-fields, the fact is that all of them can be considered as an application of statistical estimators.
Even though the research topic spreads across different sub-fields, a common thread that bind them all together is the application and development of statistical estimators.
For single channel SAR, it is the estimation of underlying back-scattering coefficients being corrupted with stochastic noise.
In the case of multi-channel SAR, it is the estimation of a target's polarimetric signature, being measured in multiple channels simultaneously and stochastically.
And for the problem of target detection or classification, it is the likelihood estimation that the target belongs to a certain category.
% which is then fed into a decision rule to make a last judgement.

Thus, the main objective of this research is to derive effective statistical estimators for the reconstruction and classification of the underlying target information in speckled SAR and POLSAR data.

In order to design and evaluate effective estimators, we centre our effort on developing a good computational framework. 
The objective of such framework being:
\begin{enumerate}
\item To build, develop and validate statistical models for different stochastics processes that are inherent within the polarimetric SAR domain.
\item To allow easy, fast and repeatable simulations and experiments of various stochastic processes in designed, focused, as well as broad-based scenarios.
\item To allow easy application of different existing estimators as well as various transformations into focused and simulated scenarios as well as into practical real measured data.
\item To allow development and calculation of good performance indices for estimators. 
\item To allow visualization and analysis of the performance of different stochastic estimators.
\item Ultimately, to allow development of better and more accurate estimators, whose results can be verified: not only qualitatively on real life images, but also quantitatively through rigorous simulations.
%quantitatively in rigorious simulations and qualititatively in real measured data.
\end{enumerate}

The possible benefits of the research project are:
\begin{enumerate}
\item Practical benefits
\begin{enumerate}
\item The computational frameworks allow for faster, quicker and more focused experiments. 
Often, a statistical hypothesis can be observed or rejected by designing and implementing an appropriate computational experiment.
\item The framework allows for easier analysis, and thus more insight to be gained, on the performance of an estimator. 
This, in turn, allows for design and development of better estimators.
\item Our research to find evidence, and hopefully rigorous proofs, supporting the advantages of (partial) polarimetric SAR implementations over conventional SAR data also yields high practical interest.
%Our research to find evidence, and hopefully rigorious proofs, supporting the advantanges of A over B also yields high practical interests.
%Our intended research to address a pressing practical trade-off, between a choice of better resolution and an option of complete polarimetric information, also yields high practical interests.
Apparently the topic has also received a fair share of attention and supports from EADS \footnote{This PhD research project is run in conjunction with EADS and has an EADS co-supervisor}.
\item The investigative research of ship monitoring using polarimetric SAR satellites is also apparently of high interest to EADS, as evidenced by their own in-house investigations.
\end{enumerate}
\item Academic benefits
\begin{enumerate}
\item The computational framework allows for quantitative evaluation and comparison of estimators.
With fast and repeated experiments, the framework allows one to not only compute a single performance value on a single experiment, but also to analyze the whole behaviour of such performance measures over repeated stochastic simulations.
\item For the sub-topic of single-channel SAR imagery, we believe an estimate for the PDF of the true underlying coefficient will be an unique research result.%is the first of such kind.
\item For the multi-channel POLSAR sub-field, we believe our proposed approach in speckle filtering of the off-diagonal elements in the polarimetry covariance matrix is also a novelty.
\item For the specific application of ship detection and classification, we believe our approach of evaluating the performance of higher-resolution and partial polarimetry modes using ground truth from AIS data is also unique.
\end{enumerate}
\end{enumerate}

As a testament on the effectiveness of this research, the work discussed in this report was published as two separate papers in 2010, at ISCCSP\cite{Le_2010_ISCCSP} and ACRS\cite{Le_2010_ACRS}.

\subsection{The Research Questions}

Figure 
	\ref{fig:residual_as_noise} 
shows that: when compared with optical remote sensing images, SAR data suffers from serious degradation.
%SAR data suffers from stochastic speckle phenomena, which poses numerous challenges in human capabilities of understanding SAR images.
%Years of research in SAR signal processing has resulted in various speckle filters, all of which aims to produce better visually de-speckled gray images from the degraded SAR data.
The degraded SAR data almost invariably requires a treatment of speckle filtering to yield more comprehensible images.
Recent advances in Polarimetric SAR (POLSAR) have produced multiple channels of SAR data, which are conceptually similar to the multiple channels found in colourful optical images.
However, the stochastic effects of speckle on multivariate and correlated data is ever more eminent.
The key proposed research question is how to reconstruct clean, colorful visualization of Polarimetric Information from the new but speckled PolSAR satellites.
Towards the end of the research project, an optional extended question shall be how this extra de-speckled polarimetric information may help in the tasks of target detection and classification. 


The main research question is conceptually divided into the following subsidiary questions to investigate:
% and project phases include. 
%1. How to simulate SAR and Polarimetric SAR data (quick pass, no lit review)
\begin{enumerate}
\item For single channel SAR data, the main aim would be how best to produce and quantitatively evaluate the de-speckled gray images. 
The main contributions could be that: for the first time, the PDF for the underlying back-scattering coefficient appears to be estimate-able, at least in the log-transformed domain. 
It is hoped this new knowledge may help in developing better speckle filters, especially those filters of \textit{Maximum a Posteriori} (MAP) nature.
Another possible contribution could be: how the homoskedasticity in the log-transformed domain could help to consistently evaluate the performance of different speckle-filters being applied on heterogeneous spatial variations.
%2. how the PDF of underlying back-scattering coefficient, in the log-domain, could be estimated and how this could help in improving the performance of the Maximum A Posteriori class of speckle-filters.
%The topic and the concept itself is a new topic
\item For multi-channel SAR data, the main aim would be to produce a smoothed and colourful visualisation of Polarimetric information. 
The main contributions would be a methodology to extends the results of single-channel SAR speckle filtering towards the issue of estimating the off-diagonal elements in polarimetric covariance matrix. 
Another fundamental contribution could be: how to model, simulate and visualize the multivariate and correlated channels in POLSAR data.
\item The third portion of this project is to investigate how full polarimetric information could be reconstructed from partial polarimetry modes. 
The main aim of this section is to fill the gap between practical dual-polarimetric implementations and theoretical established full polarimetric results. 
%The practical application is synchronized with current satellite hardware deployments and trade-offs.
\item An optional collaborative portion is to investigate how the extra information in different polarimetric modes may help in the task of target detection and classification. 
The main aim of this stage is to demonstrate the capability of (partial) polarimetric information over traditional single-channel SAR data.
%and also to evaluate the trade-off between higher resolution offered by partial polarimetric architecture and the more complete full polarimetric mode.
%Algorithms may have been and will also be proposed to tackle these issues, but the need to evaluate the algorithms requires rigorious ground-truthed data.
\end{enumerate}

\subsection{Research Methodology}

As the speckle phenomena is stochastic in nature, speckle filtering has been put into the framework of statistical estimation.
This project's research methodology will be based extensively on a statistical estimation framework. 
The framework consists of three stages:

\begin{enumerate}
\item Stage 1: data statistical analysis. 
We aim to propose a preferred data transformation and decomposition that may allow applications of existing computer science techniques. 
The statistical models built after such a transformation are validated against real-life data.
\item Stage 2: predictor and estimator development.
Various statistics based computational intelligence and/or signal processing methods are intended to be applied. 
\item Stage 3: results analysis and performance evaluation.
Experiment results are evaluated both qualitatively on real images and quantitatively through simulated data. 
%Stochastic simulation based experiments with evaluation being both through consistent quantitative measures and visual qualititative comparisions from real captured imagery. 
\end{enumerate}

\subsubsection{Data Collection}

We have used, and plan to keep using simulated data extensively. 
There appears to have a number of advantages in using simulated experiments:
\begin{enumerate}
\item Once the model is validated against real-life data, simulated experiments can be carried out on more focused scenarios.
This allows smaller data sets to be generated and analysis can become faster and more accurate.
%Experiments can be carried out on smaller scale data, with validation from real images to come later
\item Contrary to real-life images where underlying coefficients are to be estimated, ground truth is readily available in simulated experiments.
\item Quantitative evaluation is possible not only via a single experiment on a single real-image but via repeated stochastic runs. 
\item Thus not only single values of mean or mean-squared-error is to be evaluated. 
As the whole response PDF is available, statistical bias or heteroskedasticity can also be reported.
%But also statistical bias or heteroskedasticity can be estimated as the whole pdf distributions are available. 
\end{enumerate}

Besides simulated  data, we have gathered a large amount of real-life data.
The table below lists data made available to us, courtesy of EADS Innovation Works Singapore. %(TODO:RECOLLECT)

\begin{table}
\centering
\begin{tabular}{|c|c|c|}
\hline
Radar Platform & Polarization Mode & Data Capture Time \\
\hline
RadarSat 2 & Full Quad Pol & 29 Apr 2009 \\
RadarSat 2 & Full Quad Pol & 05 Apr 2009 \\
RadarSat 2 & Full Quad Pol & 12 Mar 2009 \\
RadarSat 2 & Full Quad Pol & 23 Jan 2009 \\
RadarSat 2 & Full Quad Pol & 30 Dec 2008 \\
RadarSat 2 & Full Quad Pol & 06 Dec 2008 \\
RadarSat 2 & Full Quad Pol & 19 Oct 2008 \\
RadarSat 2 & Full Quad Pol & 12 Nov 2008 \\
RadarSat 2 & Full Quad Pol & 25 Sep 2008 \\
TerraSar-X & Dual Pol (HH-VV) & 18 Apr 2008 \\
NASA-JPL AirSar & Full Quad Pol & 2001 \\
\hline
\end{tabular}

\label{tbl:collected_data}
\caption{Collected imagery data}
\end{table}

Furthermore, for the task of ship detection and classification using polarimetric data, we intended to capture ship's location using readily available Automatic Identification System (AIS) systems. 
The data collection strategy will be synchronized tasks of satellite imagery and AIS data capture.
AIS data, in a close-to-port environment, will be used as ultimate ground truth to evaluate different target detection and classification algorithms.

\subsubsection{Result Interpretation and Evaluation}

Simulated experiments allows evaluation of estimators against ground-truth.
Errors can be quantitatively analyzed and methods to reduce such errors can be developed.
Thus simulated experiments could produce quick results, which may provide faster analysis and further insights for developing more accurate estimators.

Practically in the end, all estimators would always need to be applied on real images.
Contrary to experiments on simulated data, experiments on real images can only be compared qualitatively.
This however, is critical and will always be used in our investigations.
The reason is that this helps to guard against any systematic errors in the used assumptions and approximations made in developing the theoretical models.
Towards that end, real-data can help to build and validate statistical models.

\subsubsection{Theory Development and Research Strategy}
For single channel SAR data, we propose the use of a log transform to evaluate speckle filter's performance. 
We have proved that homoskedasticity property is indeed manifested in the log-domain.
The traditional performance index in comparing speckle filters, i.e. the Equivalent Number of Looks (ENL), has also been shown to be related and calculable from the standard Mean Squared Error (MSE) inside the log-domain.
We are working to prove that, with the homoskedastic property asserted, the universal Mean Squared Error can be a good performance indicator, even on heterogeneous spatial variations.

Furthermore, as the point spread function (PSF) is available in the log-transformed domain, it appears that, for the first time, the histogram of the underlying back-scattering coefficient may become estimate-able. 
In figure 
	\ref{fig:noise_pdf_as_psf:conv_inputs}, 
the underlying single channel gray image histogram is presented along with the noise PDF, in the log-transformed domain. 
As the noise is additive, it is expected that the noise PDF will act as a point spread function.
This makes the observable speckled image histogram mathematically equivalent to the convolution of the original image's histogram and the noise PDF.
This fact is evidenced in figure 
	\ref{fig:noise_pdf_as_psf:conv_output_observable}, 
where except for the un-calibrated shifting, the shapes of theoretical convolution and observed histogram are reasonably similar.
We hypothesize that: given the observable SAR data and its histogram, it should be possible to estimate the original underlying signal's histogram through some regularized deconvolution methods.
We intended to start by applying the Richardson-Lucy algorithm \cite{Richardson_1972_JOptSocAm, Lucy_1974_JAstronomical} for such estimations.

\begin{figure}[h!]
\centering
	\subfloat[Original image's histogram and PSF]{
		 \epsfxsize=3in
		 \epsfysize=3in
 		 \epsffile{images/log_PSF_NTU_gray_histogram.eps}
		 \label{fig:noise_pdf_as_psf:conv_inputs}
	} 
	\hfill
	\subfloat[Convoluted and speckled image histogram]{
		 \epsfxsize=3in
		 \epsfysize=3in
		 \epsffile{images/log_convoluted_NTU_speckled_histogram.eps} 	
		 \label{fig:noise_pdf_as_psf:conv_output_observable}
	}
\caption{Observable convolution of PSF}
\label{fig:noise_pdf_as_psf}
\end{figure}
  
For multi-channel SAR data, we focus on partial polarimetric SAR satellite data.
Attempts have been made to model the data, especially the off-diagonal covariance matrix elements, i.e. $S_hS_v^*$.
However, the models are overly complicated which made proposed PolSAR speckle suppression algorithms became overly complex (see polarimetric data modelling section in Chapter 2).
We preferred an alternative approach.
We propose to transform the polarization basis of dual-polarized SAR data to help estimate this element.

We first pick up a recent revelation by Raney, 2006 \cite{Raney_2006_IGARSS}, that the back scattering waves' polarization captured in dual-pol data can be completely described using Stokes parameters.
%From the usually given pairs of $S_h$ and $S_v$, Raney has shown that Stokes parameters can be computed.
Furthermore, Stokes parameters can be measured in multiple polarization basis, which can also be synthesized from the given data (see polarization basis transform section in Chapter 2).
%Both transformations, i.e. from $\{h,v\}$ to circular basis $\{l,r\}$ and from $\{h,v\}$ to $\{+\pi/4,-\pi/4\}$ modes, are special cases of the transformation formula.
That is:
	the measured signal at $\pi/4$ linear polarizer as: $\xi_+ = \frac{\xi_h + \xi_v}{\sqrt{2}}$,
	at $-\pi/4$ linear polarizer as: $\xi_- = \frac{\xi_h - \xi_v}{\sqrt{2}}$, 
	at left circular polarization as: $\xi_l = \frac{\xi_h + i \xi_v}{\sqrt{2}}$, 
	and at right circular polarization as $\xi_r = \frac{\xi_h - i \xi_v}{\sqrt{2}}$.

The Stokes vectors, when aligned at these transformed projections are given by the following well-known equation:
\begin{equation}
S_t = \left(
\begin{array}{c}
 |\xi_h|^2 + |\xi_v|^2 \\
 |\xi_h|^2 - |\xi_v|^2 \\
 2 \Re( \xi_h \xi_v^* ) \\
 2 \Im( \xi_h \xi_v^* ) 
\end{array}
\right)
= \left(
\begin{array}{c}
 |\xi_+|^2 + |\xi_-|^2 \\
 -2 \Re( \xi_+^* \xi_- ) \\
 |\xi_+|^2 - |\xi_-|^2 \\
 2 \Im( \xi_+^* \xi_- ) 
\end{array}
\right)
= \left(
\begin{array}{c}
 |\xi_l|^2 + |\xi_r|^2 \\
 2 \Re( \xi_l^* \xi_r ) \\
 -2 \Im( \xi_l^* \xi_r ) \\ 
 |\xi_l|^2 - |\xi_r|^2 
\end{array}
\right)  
\end{equation}

Evidently, the real and imaginary parts of $S_hS_v^*$ is closely related to Stokes Parameters.
Specifically they may be considered as the subtraction of two random SAR intensity variables, e.g., $|S_l|^2 - |S_r|^2$. 
%Since ShSh is exponential this can be calculated through convolutions, if each channels are independent. 
%Laplace distribution is the difference between two identical exponential distributions. 
For the same traditional reasoning of $|S_h|^2$ and $|S_v|^2$, we hypothesize that all the four intensities calculated from the transformation of polarization basis, i.e., $|S_+|^2$, $|S_-|^2$, $|S_l|^2$, $|S_r|^2$, follow negative exponential distributions. 
The hypothesis is supported by our preliminary experiment results, where we synthesize the basis transformed intensity from real-measured polarimetric data.
It then follows that $S_hS_v^*$ can be despeckled by independently de-speckling the four synthesized intensities.
Read chapter 4 for our work in progress in single-channel SAR speckle filtering.
We hope to go further speculating that: the speckle-filtering methodologies in dual-polarization SAR data is extensible to full polarimetric SAR data.

%In fact, we could build and validate statistical model for all elements of POLSAR data.
%One steps further and we could demonstrate the methodology for target detection and classification in dual-pol data.
%The original results in reconstruction from CP to FP could be revised and evaluation may be on estimates of Entropy, Polarization angle, Anisotropy (Cloude).

%We could also go further and build a model for simulation of POLSAR images.
%Stokes matrix can uniquely describe covariance matrix (CP). Thus dual-pol data could be visualized as a color image, and a color image could be used to simulate CP data? (Simulation). Math: stokes vector to degree of polarization, and two angles! map that to HSV colors.
%This study would also help to identify pdf() rules for polarized light parameters being: degree of polarization, polarization angles. On way to remove speckle is multi-look processing. The effects of multi-look as biased estimation of these parameters will be discussed. The advanced way is bias-correction. Biasness can be investigated by simulation. A more prolonged study is to see the simulate pdf() (additive!) and match that with real data for model identification (ignore this!). Alternatively statistical analysis of off-diagonal Covariance matrix elements. Speckle suppression algorithms can be developed. Evaluation is hard at this point.
%Should a review of target decomposition is needed see (Unal,2009,PIER).
%One can change polarization basis to circular polarizations (see Boulbry,2007,ProcSPIE). 
%decomposition and compact polarimetric target decomposistion

To demonstrate the use of partial polarimetry for target detection and classification tasks, we could proceed with either approach.
The first approach being: first full polarimetric information is to be reconstructed from partial polarimetry data.
Then standard full polarimetric target detection and classification techniques can be applied on synthesized data.
Read Chapter 3 for our published work in extending the reconstruct-ability of full polarimetric information from partial polarimetry data to urban and man-made areas.
The second approach would be to invent a totally new target detection algorithm making use of the fact that partial polarimetric data also contains certain basic information about the target's polarimetric signature (see partial polarimetry section in Chapter 2).

\subsection{Theoretical and Practical Contributions}

For SAR data, the main focus is on niches areas like evaluation of filters, estimate of PDF, new filter?
For PolSAR data, the main focus is more fundamental: investigation and modelling of PolSAR data channels. 
The approach is to start from simulation and ENL estimation in multi-variate extension of Gauss Markov theorem, 
new filters can be developed, initially with limited assumptions in modelling, and is evaluated against an evaluation protocol.
Assumptions are made and tested in the process of evaluation of PolSAR speckle filter. 
Valid assumptions are then checked against mathematical formula and transformations.

The methodology in analysing multi-channel Polarimetric SAR data is to based on target decompositions.
The multi-variate data is decomposed into conceptually and physically independent quantities and analysis is conducted on them individually!
Another possibility is to find a consistent distance measure from the multi variate, multi-dimensional SAR data.
This distance measure would be useful in the task of target detection from a randomized background.
It could also be useful in land cover classification as well as change detection fields of PolSAR

We believe that the main focus of this research, partial polarimetric speckle filtering, has the following advantages:
\begin{enumerate}
\item The topic and the concept itself are newly developed and have just recently being introduced into the community.
\item Compared to full polarimetric data, partial polarimetry is much more compact.
Yet it still exhibits the same multi-variate correctional statistics properties. 
\item The practical application is synchronized with current satellite hardware deployments and trade-offs.
\end{enumerate}

The possible benefits of our proposed research project are:
\begin{enumerate}
\item Practical benefits
\begin{enumerate}
\item The computational frameworks allow for faster, quicker and more focused experiments. Often a point can be proven or rejected by designing and implementing an appropriate computational experiment.
\item The framework allows for easier analysis and allows for more insight to be gained in an estimator's performance. This in turn allows for the development of better estimators.
\item Our research topic for partial polarimetric SAR has high practical interest, addressing a pressing practical trade-off between the better resolution in partial modes versus the more complete but coarser resolution in theoretical full polarimetry.
Apparently the topic also generated practical interest from industry, i.e. from EADS.
\item The investigative research of ship monitoring using polarimetric SAR satellites is also apparently of high interest to EADS, as they are themselves working in this area.
\end{enumerate}
\item Academic benefits
\begin{enumerate}
\item The computational framework allows for quantitative evaluation of estimators.
And with fast and repeated experiments, the framework allows the computation of not only a single performance counter value on a single experiment, but also allow the full distribution of such performance over repeated stochastic simulations to be inspected.
\item For our research topic in the SAR community, we believe that obtaining an estimate of the true underlying PDF is a first of such systems.
\item For multi-channel POLSAR, we believe our approach to tackle speckle filtering of off-diagonal elements in polarimetric covariance matrix is also a novelty.
\item In the topic of target (ship) detection and classification, we believe our approach of evaluating the trade off between higher-resolution and partial polarimetry using AIS as ground truth is also unique.
\end{enumerate}
\end{enumerate}


	\section{Organisation of the Thesis}
The rest of this thesis is organized into 5 chapters, as follows:
The next chapter will survey the related publications
  and state-of-the-art techniques in modeling, processing, filtering and evaluating (POL)SAR speckle filter will be presented.
It will also discuss the different measures of distance for both SAR and POLSAR data, together with their extensive applications. 
Through these discussion, the need for a homoskedatic and scalar model for the multivariate and heteroskedastic polsar is brought forward.

The heart of the thesis lies in chapters \ref{chap:sar} and \ref{chap:polsar}.
While chapter \ref{chap:sar} presents the initial homoskedastic model for SAR data, 
chapter \ref{chap:polsar} illustrates the more recent and more generic model for POLSAR.
Chapter \ref{chap:polsar} also specifically describes how the initial model for SAR can be considered as a special case of the POLSAR model.

Chapter \ref{chap:applications} presents several of our published techniques in handling both SAR and POLSAR data.
Theses techniques can all be considered as applications of the theoretical model presented in chapters \ref{chap:sar} and \ref{chap:polsar}.
Demonstrated applications include, but the possibilities are not limited to: clustering algorithms, speckle filtering techniques or a novel methodology to evaluate (POL)SAR speckle filters.
Finally, the contribution of the thesis is discussed in the conclusional chapter \ref{chap:conclusions}.
